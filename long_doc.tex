% A modified copy of my NDSEG Research Proposal
\documentclass[10pt]{article}

\usepackage[lastname=Lewis,title={Research Proposal}]{styles/cannonprose}

\begin{document}

\thispagestyle{empty}

\title{Research Proposal}
\author{W. Cannon Lewis II}

\maketitle

\section*{Motivation}

Intelligent robots such as autonomous transports, unmanned aerial vehicles,
and small, agile ground vehicles will be key to the future of combat. In the
near term, these technologies enable enhanced logistics and surveillance without
endangering human lives, and hence dramatically expand the options, abilities,
and safety of soldiers on the ground. Recent advances in artificial intelligence
and machine learning have driven a surge of public interest in intelligent
systems such as self-driving cars. However, the companies currently developing
self-driving cars and other consumer robotics benefit from strong modeling
\linedmarginnote{This is a test margin note on the first page. It sticks out here, but
aligns below.}
assumptions, such as the presence of paved roads and traffic regulations, that
often do not apply in military applications. The environments that military
robots must navigate are frequently unmodeled, unstructured, and uncertain, from
the relatively simple task of rapid off-road navigation to persistent
aerial surveillance of a unmapped towns. Thus, the Army has need of intelligent
systems that are capable of adapting to unknown or changing conditions.
Furthermore, since Army robots are expected to be deployed in mixed-autonomy
teams composed of soldiers and heterogeneous robotic systems,
the desired intelligent systems must be explainable and amenable to human
intervention. These twin problems motivate my research: we need robots that are
capable of adapting to unforeseen circumstances, but which do so in ways that
allow for human intervention.

\section*{Core Research Idea}
As a motivating example, consider the problem of maneuvering an agile ground
vehicle in high-slip, muddy terrain. This problem is representative of a broad
class of open problems: that of navigation in an environment whose dynamics are
defined by continuous, differential dynamics with external disturbances. These
environments are difficult to model precisely, which places them largely outside
the purview of existing methods for motion planning.
Existing methods for motion planning tend to assume that a robot's environment
will obey simple, predictable dynamics, but problems such as agile navigation do
not satisfy such assumptions. I propose to use theoretically motivated
Reinforcement Learning to efficiently learn human-understandable robot policies
that overcome these challenges.

Reinforcement Learning (RL) has been shown to be a powerful tool for learning
about ongoing interactions between a robot and its
environment, but much of the existing theory
supporting RL only applies to problems with discrete state and action
spaces, such as navigation on a grid.
Reinforcement learning methods have been successfully applied to a variety of
operations research and cyberphysical systems problems,
but have not been broadly deployed for fundamentally high-dimensional and
continuous robotic systems. Though it has been suggested that Deep RL capably
extends RL to continuous spaces, my paper in submission to the 2019
International Conference on Robotics and Automation (ICRA) suggests that Deep RL
is not yet empirically capable of solving even simple robotic problems. 

All is not lost, though; I hypothesize that insights from discrete RL can be
extended to the continuous case. I propose a re-examination of prior work in
simplifying large discrete environments, with a view toward
creating extensions of this theory to continuous state and action spaces. I will
develop both theoretic and empirical demonstrations that it is possible to learn
\linedmarginnote{Here's a math margin note: $y = Ax + b$}
simple representations amenable to discrete RL from continuous environments. In
this way, the efficiency and theoretical guarantees that discrete RL enjoys can
be extended to fundamentally continuous problems. This discrete representation
generation will also make robot behavior more transparent to human supervisors
and other users of my proposed system. If I instead find that this is not
possible, my work will open up new avenues of research in continuous RL as I
explore how these problems can be solved without discrete tools.

More precisely, the starting point for my research will be to use funnel
libraries to discretize a robot's environment using a
  model of the true dynamics. This is related to existing work that
seeks to reason over funnels using formal logic,
and to prior work in the control of cyber-physical
systems, but I will instead use discrete RL
algorithms in the generated discrete space to learn an approximately optimal
\emph{discrete} policy. Note that this discrete policy induces a
\emph{continuous} policy in which a single controller is used for all
continuous states corresponding to one discrete state. This part of my work will
extend prior work on RL by giving theoretical bounds on the sub-optimality of
the induced continuous policy. I will then expand the scope of my research by
investigating how the induced continuous policy may be used to change the
discrete representation (e.g., by making the discrete representation more
fine-grained near high reward regions). These steps of policy learning and
discretization adjustment could then be run in a loop, which I will seek to
prove converges to a near-optimal policy. I hypothesize that, as the discrete
representation adapts to the dynamics that the robot perceives, it may become
more efficient for discrete RL and more meaningful to human observers. I will
seek to prove this hypothesis both theoretically and empirically using my full
framework.

Note that my framework exposes an important opportunity for human cooperation
and intervation in its intermediate stage, before representation generation and
policy learning are run in a loop. Semi-autonomous learning may be desirable in
many near term applications of robotic autonomy, such as off-road navigation of
supply transport vehicles. In my framework, a human user could easily specify a
domain-specific discretization of a continuous task using prior knowledge,
which would then be used for discrete RL as described previously. In this way, human
\linedmarginnote{This is a margin note on a non-title page. It fits neatly within the
page, and is offset slightly from the outside edge.}
knowledge can be used to bootstrap learning or safeguard against learning
failure in time- or safety-critical applications. Even in the full version of my
framework, human-provided discretizations could be used to ``warm-start''
learning. These opportunities for human-tunable learning clearly align well with
the Army's near term Robotics and Autonomous Systems (RAS) strategic
priorities for semi-autonomous systems.

\section*{Evaluation of Success}
In addition to the theoretical contributions that I intend to make to the
continuous RL literature, I intend to assess the success of my method
empirically through evaluation on a series of increasingly realistic
environments. I will first test my method on simple simulated models of
differential drive vehicles in environmental dynamics defined by differential
equations with external disturbances. I will then leverage my lab's extensive
experience with robotic simulation to test my method in more realistic
environments defined in physics simulators such as Gazebo. Finally, I will test
my method on real robots by controlling Rice's aerial drones or my group's UR5
and Fetch robots. This series of tests will establish the success of my method
on a broad class of continuous problems with external disturbances, and show
the applicability of my proposed method to robotics in the real world.

\section*{Army Impacts}
This proposal is intended to respond directly to the U.S. Army Broad Agency
Announcement (BAA) W911NF-17-S-0002-03. More specifically, this
proposal falls under section II.A.1.c.i.(4): \emph{Intelligent Systems}. My
research aims to address the criticism that machine learning for robotics
``still lacks the rigor, agility, and adaptation'' necessary for deployment
in changing or unknown environments. The BAA calls for learning systems that
``can continuously learn and evolve to changing context and environment'' and
``deal with incorrect input.'' My proposed research is a prime example of such a
system; the RL setting is well-poised for online learning and allows easy
incorporation of probabilistic noise models or error models. Furthermore, as
detailed above, my framework could achieve new levels of robustness by
integrating with human operators in- or on-the-loop of learning. In an
adversarial setting where a combatant attempts to misdirect learning, a human
could intervene in the representation generation portion of my framework to
prevent adverse effects. Thus, my proposed research clearly advances the Army's
interest in advanced machine learning and intelligent systems.

\section*{Service and Societal Impacts}
My proposed research will result in a new framework for autonomous behavior in
unmodeled continuous environments. This framework has the potential to
dramatically expand the capability of Army robots to perform in complex
environments. Looking beyond Army interests, my method
could support autonomous systems which magnify first responders' abilities in
the face of natural disasters and other emergency settings. My research will
contribute to the fundamental scientific literature on continuous RL and
establish new theoretical guarantees for the use of RL in robotics, paving the
way for further developments in this area. I will also engage with the public
through science communication initiatives such as publishing popular science
articles, conducting outreach as the host of the Rice Computer Science podcast,
and encouraging popular engagement with STEM by developing outreach programs
that focus on creativity and free exploration. Finally, I am committed to making
all of the empirical analysis that I conduct during my research open source and
easily reproducible by continuing to follow industry best practices, such as
packaging all research code in Docker containers which can be easily shared.

\end{document}
